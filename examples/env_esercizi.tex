% -*- ConText -*-
% It specifies the format of head named 'title'
% Specifically the style of the font: sans serif
% + bold + big font.

\startenvironment env_esercizi
\mainlanguage[italian]
\setuppapersize[A4]
\setuplayout[
  topspace=1cm,
  backspace=2cm,
  width=middle,
  height=middle,
  header=0pt]

\setupbodyfont[10pt]

\setupunit[method=5]

%\setupthinrules[before={}]
\def\compito[#1]
 {\getrawparameters[Compito]
 [title=,#1]}

\def\makecompitotitle{%
\page%
\doif{\Compitoscuola}{alberghiero}{%
\setuplayout[header=1.5cm]
\setupheadertexts[{\framed[frame=off,width=broad]{\externalfigure[img/loghi/alberghiero2.png][height=1.5cm]\hfill\externalfigure[img/loghi/alberghiero1.png][height=1.5cm]\hfill \externalfigure[img/loghi/alberghiero3.png][height=1.5cm]}}]}

\startalignment[center]
  \blank[force,1*small]
      {\tfc \Compitotitle}
  \blank[1*small]
\stopalignment
\setuppagenumber[number=1]
\resetesercizio
\resetsoluzione

}

\setuphead[title][style={\ss\bfd},alternative=middle]

\def\mypagenumber#1%
  {Pagina #1 di \lastpagenumber} % \totalnumberofpages

\setuppagenumbering[location=footer, command=\mypagenumber]

\def\rfoot{}
\setupfootertexts[][\rfoot]

\setuphead[section][style={\ss\bfb}]
\setuphead[subsection][style={\ss\bfa}]

\setupquotation[style=italic]

\def\infoform{%
\starttabulate[|lw(.5\textwidth)|lw(.5\textwidth)|]%
  \NC Nome: \fillinrules \NC Classe: \fillinrules \NC \NR%
  \NC Cognome: \fillinrules \NC  Data: \fillinrules \NC \NR%
\stoptabulate%
}

%%%%%%%%%%%%%%%%%%%%
%% esercizio soluzione
\definemode[soluzioni][keep]
\definemode[esercizio][keep]
\defineenumeration[esercizio][text={},stopper={.},alternative=left,width=2ex]% stopper or right coupling=soluzione
\defineenumeration[soluzione][text={Soluzione},stopper={.},alternative=left]% stopper or right ,coupling=esercizio
\defineblock[soluzione]
%\keepblocks[soluzione]
\hideblocks[soluzione]
\def\printsoluzioni{\page \subject{Soluzioni} \useblocks[soluzione]}
%%%%%%%%%%%%%%%%%%%%
%% some env
\definestartstop[parts][before={\blank[small]\startitemize[i,joinedup][stopper=.]},
                        after={\stopitemize}]  
\definestartstop[verofalso][before={\startitemize[a,packed][stopper=\thinspace)]
\let\olditem\item\define\item{\olditem \framed[width=1em,strut=yes]{V}\thinspace\framed[width=1em,strut=yes]{F}\thinspace}},
                        after={\stopitemize}]

\definestartstop[truefalse][before={\startitemize[a,packed][stopper=\thinspace)]
\let\olditem\item\define\item{\olditem \framed[width=1em,strut=yes]{V}\thinspace\framed[width=1em,strut=yes]{F}\thinspace}},
                        after={\stopitemize}]

\definestartstop[choices][before={\blank[small]\startitemize[A,packed,joinedup]},after={\stopitemize}]


%%%%%%%%%%%%%%%%%%%%
%% RSFS font 
 % \font\tenscr   = rsfs10 at 10pt %bodyfontsize at 12
 % \font\sevenscr = rsfs7  at 7pt  %scriptfontsize at 9
 % \font\fivescr  = rsfs5  at 5pt  %scriptscriptfontsize

 % \skewchar\tenscr   = '127   %'177
 % \skewchar\sevenscr = '127
 % \skewchar\fivescr  = '127

 % \newfam\scrfam

 % \textfont\scrfam         = \tenscr
 % \scriptfont\scrfam       = \sevenscr
 % \scriptscriptfont\scrfam = \fivescr

 % \def\scr{\fam\scrfam}

%%%%%%%%%%%%%%%%%%%%%%%%%%%%%%%%%%%%%%%%%%%%%%%%
%% METAPOST
\startMPinclusions
  input metaobj;

  vardef markanglebetween (expr endofa, endofb, common, length, str) =
        save curve, where ; path curve ; numeric where ;
        where := turningnumber (common--endofa--endofb--cycle) ;
        curve := (unitvector(endofa-common){(endofa-common) rotated (where*90)}
        .. unitvector(endofb-common)) scaled length shifted common ;
        draw thefreelabel(str,point .5 of curve,common) withcolor black ;
       curve
  enddef ;

  u=5mm;
  def grid(expr xi,xf,yi,yf)=
    for i=xi+1 upto xf-1:
      
      draw (i*u,yi*u)..(i*u,yf*u) withcolor .7white ;
    endfor
    for i=yi+1 upto yf-1:
      draw (xi*u,i*u)..(xf*u,i*u) withcolor .7white;
    endfor
  enddef;
  
  def assi(expr xi,xf,yi,yf)=
    drawarrow (xi*u,0)--(xf*u,0);
    drawarrow (0,yi*u)--(0,yf*u);
    label.bot(btex $x$ etex, (xf*u,0));
    label.lft(btex $y$ etex, (0,yf*u));
  enddef;
  
  def labelgriglia(expr xi,xf,yi,yf)=
    for i=xi+1 upto xf-1:
      draw (i*u,-2pt)--(i*u,2pt);
      draw textext(decimal(i)) scaled .8 shifted (i*u-3pt,-5pt);    
 %     label.llft(decimal(i) infont "ptmr" scaled (6pt/fontsize defaultfont), (i*u,0));
    endfor
    for i=yi+1 upto yf-1:
      draw (-2pt,i*u)--(2pt,i*u);
      draw textext(decimal(i)) scaled .8 shifted (-6pt,i*u+4pt);      
%      label.llft(decimal(i) infont "ptmr" scaled (6pt/fontsize defaultfont), (0,i*u));
    endfor;

  enddef;
  def righello(expr strt, stp, btick, stick, meas)=
    draw (-5+strt*u,0)--(-5+strt*u,-1cm)--(stp*u+5,-1cm)--(stp*u+5,0)--cycle;
    for i= strt step btick until stp:
      draw (i*u,0)--(i*u,-0.3cm);
      label.bot(decimal i,(i*u,-0.3cm));
    endfor
    for i=strt step stick until stp:
      draw (i*u,0)--(i*u,-0.2cm);
    endfor;
    label(btex mm etex,((strt+stp)*u/2,-0.8cm));
    draw (strt*u,3)--(meas*u,3) withpen pensquare yscaled 3pt;
enddef;

\stopMPinclusions
%%%%%%%%%%%%%%%%%%%%%%%%%%%%%%%%%%%%%%%%%%%%%%%%
%% esercizi
\startluacode
require("../esercizi")
\stopluacode

\def\var[#1][#2][#3]{\ctxlua{u.#1 = {value=[==[#2]==],svalue='#2',unit=#3}}}
\def\dervar[#1][#2][#3]{\ctxlua{u.#1 = u.defdervar(#2,#3)}}
\def\dervarp[#1]#2{\ctxlua{u.printvar(u.#2,#1)}}




\def\varp#1{\ctxlua{context.unit(u.#1.svalue .. u.#1.unit)}}
\var[g][9.8]["meter per square second"]
\var[bigG][6.67e-11]["newton square meter per square kilogram"]

\stopenvironment
