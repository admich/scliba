
\usepath[../..]
\project didattica
% \enablemode[soluzioni]


\setupbodyfont[12pt]

\starttext
\starttext

\compito[title=Verifica di fisica,scuola=none]
\makecompitotitle
\def\rfoot{ ott. 2016 (W43) f0}
\setupTABLE[frame=off]
\setupTABLE[c][1][width=0.5\textwidth]
\setupTABLE[c][2][width=0.5\textwidth]
\bTABLE
\bTR \bTD Nome: ~\hrulefill ~\eTD \bTD Classe: ~\hrulefill ~\eTD \eTR \bTR \bTD Cognome: ~\hrulefill ~\eTD \bTD Data: ~\hrulefill ~\eTD \eTR 
\eTABLE
\startframedtext[location=middle, bodyfont=8pt]
Punteggio min. 1pt. Esercizi 1--6 0.5pt. Esercizi 7--9 2pt. TOT: 10pt
\stopframedtext
\startesercizio
Una piattaforma ruota con velocità angolare costante. Fra due punti su di essa a diversa distanza dall'asse di rotazione, quale ha accelerazione centripeta maggiore? 
\startitemize[A,packed,joinedup]
\item Hanno la stessa accelerazione. 
\item Quello più distante. 
\item Quello più vicino. 
\item Non si può rispondere. 
\stopitemize
\beginsoluzione
\startsoluzione
B

 Poiché $a_c = \omega^2 R$ l'accelerazione centripeta se la velocità angolare rimane costante è direttamente proporzionale al raggio $R$, quindi aumenta all'aumentare del raggio di rotazione. 
\stopsoluzione
\endsoluzione
\stopesercizio
\startesercizio
Un punto si muove di moto circolare uniforme e impiega $\text{1,20}\ {\rm s}$ a descrivere un angolo di $\text{45,0}{\rm °}$ . Qual è il periodo del moto?
\startitemize[A,packed,joinedup]
\item $\text{0,150}\ {\rm s}$
\item $\text{4,80}\ {\rm s}$
\item $\text{1,20}\ {\rm s}$
\item $\text{9,60}\ {\rm s}$
\stopitemize
\beginsoluzione
\startsoluzione
D

 
\stopsoluzione
\endsoluzione
\stopesercizio
\startesercizio
Un auto percorre una pista circolare a velocità costante. Che cosa succede all'accelerazione centripeta se la velocità diventa il doppio? 
\startitemize[A,packed,joinedup]
\item raddoppia 
\item diventa la metà 
\item diventa quattro volte più grande 
\item non si può rispondere poiché non è noto il raggio della pista. 
\stopitemize
\beginsoluzione
\startsoluzione
C

 
\stopsoluzione
\endsoluzione
\stopesercizio
\startesercizio
I vecchi vinili esistevano di due tipi, gli LP che giravano a 33 giri al minuto e i singoli che giravano a 45 giri al minuto.Quali avevano velocità angolare maggiore? 
\startitemize[A,packed,joinedup]
\item i primi 
\item i secondi 
\item hanno la stessa velocità 
\item non si può rispondere non conoscendo il raggio dei dischi 
\stopitemize
\beginsoluzione
\startsoluzione
B

 
\stopsoluzione
\endsoluzione
\stopesercizio
\startesercizio
Lancio una palla da terra con velocità $\text{4,00}\ {\rm m/s}$ ed angolo rispetto all’orizzontale di $\text{30,0}{\rm °}$. Qual è la gittata massima raggiunta?
\startitemize[A,packed,joinedup]
\item $\text{1,41}\ {\rm m}$
\item $\text{0,707}\ {\rm m}$
\item $\text{3,27}\ {\rm m}$
\item $\text{2,45}\ {\rm m}$
\stopitemize
\beginsoluzione
\startsoluzione
A

 
\stopsoluzione
\endsoluzione
\stopesercizio
\startesercizio
Un corpo si muove su una circonferenza di moto circolare uniforme con periodo $\text{8,00}\ {\rm s}$. Quanto impiega a percorre un angolo di $\text{45,0}{\rm °}$?
\startitemize[A,packed,joinedup]
\item $\text{64,0}\ {\rm s}$
\item $\text{1,00}\ {\rm s}$
\item $\text{0,0156}\ {\rm s}$
\item $\text{5,63}\ {\rm s}$
\stopitemize
\beginsoluzione
\startsoluzione
B

 
\stopsoluzione
\endsoluzione
\stopesercizio
\startesercizio
Un punto si muove su una circonferenza di raggio $\text{30,0}\ {\rm cm}$ con velocità costante descrivendo un angolo di $\text{18,0}{\rm °}$ in $\text{0,0400}\ {\rm s}$
\startitemize[i,packed]
\item Calcola il periodo del moto. 
\item Quanti giri percorre in un secondo? 
\item Calcola velocità angolare e accelerazione centripeta. 
\stopitemize
\beginsoluzione
\startsoluzione
\stopsoluzione
\endsoluzione
\stopesercizio

\doifmode{soluzioni}{\printsoluzioni}
\stoptext
\stoptext
